\documentclass[10pt]{article}
\usepackage{longtable}
\usepackage{changepage}
\usepackage{scalerel,amssymb}
\usepackage[margin=0.75cm]{geometry}
\def\msquare{\mathord{\scalerel*{\Box}{gX}}}
\usepackage{hyperref}

\begin{document}
All Virtues, Subvirtues and Passions are between -100 and 100. 

\begin{longtable}{| p{6cm} | p{3cm} | p{10cm} |}
\caption{General Character Variables}
\label{tab:General Character}\\
\hline 
\textbf{Name} & \textbf{Data Type} & \textbf{Purpose} \\
\hline
STATE\_OF\_GRACE & Boolean & Is the character in a State of Grace? \\
\hline
NAME & String & Character name \\
\hline
AGE & Integer & Character Age \\
\hline
MORTAL\_SINS\_REMAINING & Integer & What it says \\
\hline

\end{longtable}

\begin{longtable}{| p{11cm} | p{3cm} | p{5cm} |}
\caption{Virtues and Subvirtues}
\label{tab:ViceAndVirtue}\\
\hline 
\textbf{Name} & \textbf{Data Type} & \textbf{Purpose} \\
\hline
\textbf{VIRTUE\_PRUDENCE\_SUBVIRTUE\_MEMORY} & Integer & Represents the Sub Virtue \\
\hline
\textbf{VIRTUE\_PRUDENCE\_SUBVIRTUE\_UNDERSTANDING} & Integer & Represents the Sub Virtue \\
\hline
\textbf{VIRTUE\_PRUDENCE\_SUBVIRTUE\_DOCILTY} & Integer & Represents the Sub Virtue \\
\hline
\textbf{VIRTUE\_PRUDENCE\_SUBVIRTUE\_SHREWDNESS} & Integer & Represents the Sub Virtue \\
\hline
\textbf{VIRTUE\_PRUDENCE\_SUBVIRTUE\_REASON} & Integer & Represents the Sub Virtue \\
\hline
\textbf{VIRTUE\_PRUDENCE\_SUBVIRTUE\_FORESIGHT} & Integer & Represents the Sub Virtue \\
\hline
\textbf{VIRTUE\_PRUDENCE\_SUBVIRTUE\_CICUMSPECTION} & Integer & Represents the Sub Virtue \\
\hline
\textbf{VIRTUE\_PRUDENCE\_SUBVIRTUE\_CAUTION} & Integer & Represents the Sub Virtue \\
\hline
\textbf{VIRTUE\_PRUDENCE\_SUBVIRTUE\_EUBULIA} & Integer & Represents the Sub Virtue \\
\hline
\textbf{VIRTUE\_PRUDENCE\_SUBVIRTUE\_SYNESIS} & Integer & Represents the Sub Virtue \\
\hline
\textbf{VIRTUE\_PRUDENCE\_SUBVIRTUE\_GNOME} & Integer & Represents the Sub Virtue \\
\hline
\textbf{VIRTUE\_PRUDENCE\_SUBVICE\_PRECIPITATION} & Integer & Represents the Sub Virtue \\
\hline
\textbf{VIRTUE\_PRUDENCE\_SUBVICE\_INCONSIDERATION} & Integer & Represents the Sub Virtue \\
\hline
\textbf{VIRTUE\_PRUDENCE\_SUBVICE\_INCONSTANCY} & Integer & Represents the Sub Virtue \\
\hline
\textbf{VIRTUE\_PRUDENCE\_SUBVICE\_NEGLIGENCE} & Integer & Represents the Sub Virtue \\
\hline
\textbf{VIRTUE\_PRUDENCE\_SUBVICE\_CARNAL\_PRUDENCE} & Integer & Represents the Sub Virtue \\
\hline
\textbf{VIRTUE\_PRUDENCE\_SUBVICE\_CRAFTINESS} & Integer & Represents the Sub Virtue \\
\hline
\textbf{VIRTUE\_PRUDENCE\_SUBVICE\_GUILE} & Integer & Represents the Sub Virtue \\
\hline
\textbf{VIRTUE\_PRUDENCE\_SUBVICE\_FRAUD} & Integer & Represents the Sub Virtue \\
\hline
\textbf{VIRTUE\_JUSTICE\_SUBVIRTUE\_COMMUTATIVE} & Integer & Represents the Sub Virtue \\
\hline
\textbf{VIRTUE\_JUSTICE\_SUBVIRTUE\_LEGAL} & Integer & Represents the Sub Virtue \\
\hline
\textbf{VIRTUE\_JUSTICE\_SUBVIRTUE\_DISTRIBUTIVE} & Integer & Represents the Sub Virtue \\
\hline
\textbf{VIRTUE\_JUSTICE\_SUBVIRTUE\_RESTITUTION} & Integer & Represents the Sub Virtue \\
\hline
\textbf{VIRTUE\_JUSTICE\_SUBVIRTUE\_RELIGION} & Integer & Represents the Sub Virtue, negative the sub vice of Superstition \\
\hline
\textbf{VIRTUE\_JUSTICE\_SUBVIRTUE\_DEVOTION} & Integer & Represents the Sub Virtue \\
\hline
\textbf{VIRTUE\_JUSTICE\_SUBVIRTUE\_ADJURATION} & Integer & Represents the Sub Virtue \\
\hline
\textbf{VIRTUE\_JUSTICE\_SUBVIRTUE\_PIETY} & Integer & Represents the Sub Virtue \\
\hline
\textbf{VIRTUE\_JUSTICE\_SUBVIRTUE\_PATRIOTISM} & Integer & Represents the Sub Virtue \\
\hline
\textbf{VIRTUE\_JUSTICE\_SUBVIRTUE\_OBSERVANCES} & Integer & Represents the Sub Virtue \\
\hline
\textbf{VIRTUE\_JUSTICE\_SUBVIRTUE\_DULIA} & Integer & Represents the Sub Virtue \\
\hline
\textbf{VIRTUE\_JUSTICE\_SUBVIRTUE\_OBEDIENCE} & Integer & Represents the Sub Virtue, negative being the Subvice of Disobedience \\
\hline
\textbf{VIRTUE\_JUSTICE\_SUBVIRTUE\_DILIGENCE} & Integer & Represents the Sub Virtue \\
\hline
\textbf{VIRTUE\_JUSTICE\_SUBVIRTUE\_GRATITUDE} & Integer & Represents the Sub Virtue, negative being the Sub Vice of Ingratitude \\
\hline
\textbf{VIRTUE\_JUSTICE\_SUBVIRTUE\_JUST\_VINDICATION} & Integer & Represents the Sub Virtue, with the negative being the subvice of Vengefulness \\
\hline
\textbf{VIRTUE\_JUSTICE\_SUBVIRTUE\_TRUTHFULNESS} & Integer & Represents the Sub Virtue, with the negative being the subvice of Lying \\
\hline
\textbf{VIRTUE\_JUSTICE\_SUBVIRTUE\_FRIENDSHIP} & Integer & Represents the Sub Virtue \\
\hline
\textbf{VIRTUE\_JUSTICE\_SUBVIRTUE\_LIBERALITY} & Integer & Represents the Sub Virtue \\
\hline
\textbf{VIRTUE\_JUSTICE\_SUBVIRTUE\_EPIEKEIA} & Integer & Represents the Sub Virtue \\
\hline
\textbf{VIRTUE\_JUSTICE\_SUBVICE\_HUMAN\_RESPECT} & Integer & Represents the Sub Virtue \\
\hline
\textbf{VIRTUE\_JUSTICE\_SUBVICE\_MURDER} & Integer & Represents the Sub Virtue \\
\hline
\textbf{VIRTUE\_JUSTICE\_SUBVICE\_MUTLIATION} & Integer & Represents the Sub Virtue \\
\hline
\textbf{VIRTUE\_JUSTICE\_SUBVICE\_THEFT} & Integer & Represents the Sub Virtue \\
\hline
\textbf{VIRTUE\_JUSTICE\_SUBVICE\_ROBBERY} & Integer & Represents the Sub Virtue \\
\hline
\textbf{VIRTUE\_JUSTICE\_SUBVICE\_JUDGMENT} & Integer & Represents the Sub Virtue \\
\hline
\textbf{VIRTUE\_JUSTICE\_SUBVICE\_FALSE\_ACCUSATION} & Integer & Represents the Sub Virtue \\
\hline
\textbf{VIRTUE\_JUSTICE\_SUBVICE\_PERJURY} & Integer & Represents the Sub Virtue \\
\hline
\textbf{VIRTUE\_JUSTICE\_SUBVICE\_CONTUMLEY} & Integer & Represents the Sub Virtue \\
\hline
\textbf{VIRTUE\_JUSTICE\_SUBVICE\_DETRACTION} & Integer & Represents the Sub Virtue \\
\hline
\textbf{VIRTUE\_JUSTICE\_SUBVICE\_MURMURING} & Integer & Represents the Sub Virtue \\
\hline
\textbf{VIRTUE\_JUSTICE\_SUBVICE\_DERISION} & Integer & Represents the Sub Virtue \\
\hline
\textbf{VIRTUE\_JUSTICE\_SUBVICE\_MALEDICTION} & Integer & Represents the Sub Virtue \\
\hline
\textbf{VIRTUE\_JUSTICE\_SUBVICE\_USURY} & Integer & Represents the Sub Virtue \\
\hline
\textbf{VIRTUE\_JUSTICE\_SUBVICE\_ILLICIT\_ADJURATION} & Integer & Represents the Sub Virtue \\
\hline
\textbf{VIRTUE\_JUSTICE\_SUBVICE\_IDOLATRY} & Integer & Represents the Sub Virtue \\
\hline
\textbf{VIRTUE\_JUSTICE\_SUBVICE\_DIVINATION} & Integer & Represents the Sub Virtue \\
\hline
\textbf{VIRTUE\_JUSTICE\_SUBVICE\_TEMPTING\_GOD} & Integer & Represents the Sub Virtue \\
\hline
\textbf{VIRTUE\_JUSTICE\_SUBVICE\_SACRILEGE} & Integer & Represents the Sub Virtue \\
\hline
\textbf{VIRTUE\_JUSTICE\_SUBVICE\_SIMONY} & Integer & Represents the Sub Virtue \\
\hline
\textbf{VIRTUE\_JUSTICE\_SUBVICE\_SIMULATION} & Integer & Represents the Sub Virtue \\
\hline
\textbf{VIRTUE\_JUSTICE\_SUBVICE\_BOASTING} & Integer & Represents the Sub Virtue \\
\hline
\textbf{VIRTUE\_JUSTICE\_SUBVICE\_IRONY} & Integer & Represents the Sub Virtue \\
\hline
\textbf{VIRTUE\_JUSTICE\_SUBVICE\_ADULATION} & Integer & Represents the Sub Virtue \\
\hline
\textbf{VIRTUE\_JUSTICE\_SUBVICE\_LITIGIOUS} & Integer & Represents the Sub Virtue \\
\hline
\textbf{VIRTUE\_JUSTICE\_SUBVICE\_AVARICE} & Integer & Represents the Sub Virtue \\
\hline
\textbf{VIRTUE\_JUSTICE\_SUBVICE\_PRODIGALITY} & Integer & Represents the Sub Virtue \\
\hline
\textbf{VIRTUE\_FORTITUDE\_SUBVIRTUE\_MAGNANIMITY} & Integer & Represents the Sub Virtue, negative portraying pusilanimity \\
\hline
\textbf{VIRTUE\_FORTITUDE\_SUBVIRTUE\_MAGNIFICENCE} & Integer & Represents the Sub Virtue \\
\hline
\textbf{VIRTUE\_FORTITUDE\_SUBVIRTUE\_PATIENCE} & Integer & Represents the Sub Virtue \\
\hline
\textbf{VIRTUE\_FORTITUDE\_SUBVIRTUE\_PERSEVERENCE} & Integer & Represents the Sub Virtue \\
\hline
\textbf{VIRTUE\_FORTITUDE\_SUBVIRTUE\_LONGANIMITY} & Integer & Represents the Sub Virtue \\
\hline
\textbf{VIRTUE\_FORTITUDE\_SUBVIRTUE\_MORTIFICATION} & Integer & Represents the Sub Virtue \\
\hline
\textbf{VIRTUE\_FORTITUDE\_SUBVICE\_FEAR} & Integer & Represents the Sub Virtue \\
\hline
\textbf{VIRTUE\_FORTITUDE\_SUBVICE\_FEARLESSNESS} & Integer & Represents the Sub Virtue \\
\hline
\textbf{VIRTUE\_FORTITUDE\_SUBVICE\_AUDACITY} & Integer & Represents the Sub Virtue \\
\hline
\textbf{VIRTUE\_FORTITUDE\_SUBVICE\_PRESUMPTION} & Integer & Represents the Sub Virtue \\
\hline
\textbf{VIRTUE\_FORTITUDE\_SUBVICE\_AMBITION} & Integer & Represents the Sub Virtue \\
\hline
\textbf{VIRTUE\_FORTITUDE\_SUBVICE\_INANE\_GLORY} & Integer & Represents the Sub Virtue \\
\hline
\textbf{VIRTUE\_FORTITUDE\_SUBVICE\_PARVIFICENCE} & Integer & Represents the Sub Virtue \\
\hline
\textbf{VIRTUE\_FORTITUDE\_SUBVICE\_EFFEMINACY} & Integer & Represents the Sub Virtue \\
\hline
\textbf{VIRTUE\_FORTITUDE\_SUBVICE\_PERTINACITY} & Integer & Represents the Sub Virtue \\
\hline
\textbf{VIRTUE\_TEMPERANCE\_SUBVIRTUE\_SHAME} & Integer & Represents the Sub Virtue \\
\hline
\textbf{VIRTUE\_TEMPERANCE\_SUBVIRTUE\_HONESTIA} & Integer & Represents the Sub Virtue \\
\hline
\textbf{VIRTUE\_TEMPERANCE\_SUBVIRTUE\_ABSTINENCE} & Integer & Represents the Sub Virtue \\
\hline
\textbf{VIRTUE\_TEMPERANCE\_SUBVIRTUE\_FASTING} & Integer & Represents the Sub Virtue \\
\hline
\textbf{VIRTUE\_TEMPERANCE\_SUBVIRTUE\_SOBRIETY} & Integer & Represents the Sub Virtue, negative representing Drunkenness \\
\hline
\textbf{VIRTUE\_TEMPERANCE\_SUBVIRTUE\_CONTINENCE} & Integer & Represents the Sub Virtue, negative representing the subvice of Incontinence \\
\hline
\textbf{VIRTUE\_TEMPERANCE\_SUBVIRTUE\_CHASTITY} & Integer & Represents the Sub Virtue \\
\hline
\textbf{VIRTUE\_TEMPERANCE\_SUBVIRTUE\_VIRGINITY} & Integer & Represents the Sub Virtue \\
\hline
\textbf{VIRTUE\_TEMPERANCE\_SUBVIRTUE\_MEEKNESS} & Integer & Represents the Sub Virtue, negative representing the subvice of Anger \\
\hline
\textbf{VIRTUE\_TEMPERANCE\_SUBVIRTUE\_MODESTY} & Integer & Represents the Sub Virtue, negative representing the subvice of Immodesty \\
\hline
\textbf{VIRTUE\_TEMPERANCE\_SUBVIRTUE\_HUMILITY} & Integer & Represents the Sub Virtue, negative representing the subvice of Pride \\
\hline
\textbf{VIRTUE\_TEMPERANCE\_SUBVIRTUE\_EUTRAPELIA} & Integer & Represents the Sub Virtue \\
\hline
\textbf{VIRTUE\_TEMPERANCE\_SUBVIRTUE\_SPORTSMANSHIP} & Integer & Represents the Sub Virtue \\
\hline
\textbf{VIRTUE\_TEMPERANCE\_SUBVIRTUE\_DECORUM} & Integer & Represents the Sub Virtue \\
\hline
\textbf{VIRTUE\_TEMPERANCE\_SUBVIRTUE\_SILENCE} & Integer & Represents the Sub Virtue \\
\hline
\textbf{VIRTUE\_TEMPERANCE\_SUBVIRTUE\_STUDIOUSITY} & Integer & Represents the Sub Virtue, negative representing the subvice of Curiousity  \\
\hline
\textbf{VIRTUE\_TEMPERANCE\_SUBVIRTUE\_SIMPLICTY} & Integer & Represents the Sub Virtue \\
\hline
\textbf{VIRTUE\_TEMPERANCE\_SUBVICE\_GLUTTONY} & Integer & Represents the Sub Virtue \\
\hline
\textbf{VIRTUE\_TEMPERANCE\_SUBVICE\_LUST} & Integer & Represents the Sub Virtue \\
\hline
\textbf{VIRTUE\_TEMPERANCE\_SUBVICE\_FORNICATION} & Integer & Represents the Sub Virtue \\
\hline
\textbf{VIRTUE\_TEMPERANCE\_SUBVICE\_CHEATING} & Integer & Represents the Sub Virtue \\
\hline
\textbf{VIRTUE\_TEMPERANCE\_SUBVICE\_RAPE} & Integer & Represents the Sub Virtue \\
\hline
\textbf{VIRTUE\_TEMPERANCE\_SUBVICE\_ADULTERY} & Integer & Represents the Sub Virtue \\
\hline
\textbf{VIRTUE\_TEMPERANCE\_SUBVICE\_INCEST} & Integer & Represents the Sub Virtue \\
\hline
\textbf{VIRTUE\_TEMPERANCE\_SUBVICE\_CRUELTY} & Integer & Represents the Sub Virtue \\
\hline
\textbf{VIRTUE\_TEMPERANCE\_SUBVICE\_CRUDITY} & Integer & Represents the Sub Virtue \\
\hline
\textbf{VIRTUE\_TEMPERANCE\_SUBVICE\_IMMODESTY} & Integer & Represents the Sub Virtue \\
\hline
\textbf{VIRTUE\_FAITH\_SUBVICE\_INFIDELITY} & Integer & Represents the Sub Virtue \\
\hline
\textbf{VIRTUE\_FAITH\_SUBVICE\_HERESY} & Integer & Represents the Sub Virtue \\
\hline
\textbf{VIRTUE\_FAITH\_SUBVICE\_APOSTASY} & Integer & Represents the Sub Virtue \\
\hline
\textbf{VIRTUE\_FAITH\_SUBVICE\_BLASPHEMY} & Integer & Represents 
the Sub Virtue \\
\hline
\textbf{VIRTUE\_FAITH} & Integer & Represents the Virtue \\
\hline
\textbf{VIRTUE\_HOPE\_SUBVICE} & Integer & Represents the Sub Vice, positive Presumption, negative Desperation \\
\hline
\textbf{VIRTUE\_HOPE} & Integer & Represents the Virtue \\
\hline
\textbf{VIRTUE\_CHARITY\_SUBVICE\_HATRED\_OF\_GOD} & Integer & Represents the Sub Virtue \\
\hline
\textbf{VIRTUE\_CHARITY\_SUBVICE\_SLOTH} & Integer & Represents the Sub Virtue \\
\hline
\textbf{VIRTUE\_CHARITY\_SUBVICE\_ENVY} & Integer & Represents the Sub Virtue \\
\hline
\textbf{VIRTUE\_CHARITY\_SUBVICE\_DISCORD} & Integer & Represents the Sub Virtue \\
\hline
\textbf{VIRTUE\_CHARITY\_SUBVICE\_CONTENTION} & Integer & Represents the Sub Virtue \\
\hline
\textbf{VIRTUE\_CHARITY\_SUBVICE\_SCHISM} & Integer & Represents the Sub Virtue \\
\hline
\textbf{VIRTUE\_CHARITY\_SUBVICE\_UNJUST\_WAR} & Integer & Represents the Sub Virtue \\
\hline
\textbf{VIRTUE\_CHARITY\_SUBVICE\_QUARRELLING} & Integer & Represents the Sub Virtue \\
\hline
\textbf{VIRTUE\_CHARITY\_SUBVICE\_SCANDAL} & Integer & Represents the Sub Virtue \\
\hline
\textbf{VIRTUE\_CHARITY} & Integer & Represents the Virtue \\
\hline


\end{longtable}

\begin{longtable}{| p{6cm} | p{3cm} | p{10cm} |}
\caption{General Passions from Thomistica.net}
\label{tab:GeneralPassions}\\
\hline 
\textbf{Name} & \textbf{Data Type} & \textbf{Purpose} \\
LOVE & Integer (Negative being hatred positive Love) & Governs Concupisible natural Love \\
\hline
PLEASURE & Integer (Negative being Sorrow Positive Pleasure) & Governs Concupisible Pleasure (as opposed to indulging the Irascible appetite) \\
\hline
HOPE & Integer (Negative being Despair Positive Hope) & Governs Irascible natural Hope \\
\hline
DARING & Integer (Negative being fear, Positive Daring) & Governs Irascible Courage \\
\hline
ANGER & Integer(Negative being anger and postive confidence) & Governs Anger\\
\hline
\end{longtable}

\begin{longtable}{| p{6cm} | p{3cm} | p{10cm} |}
\caption{Relationship Passions}
\label{tab:RelationshipPassions}\\
\hline 
\textbf{Name} & \textbf{Data Type} & \textbf{Purpose} \\
LOVE & Integer (Negative being hatred positive Love) & Governs Concupisible natural Love \\
\hline
PLEASURE & Integer (Negative being Sorrow Positive Pleasure) & Governs Concupisible Pleasure (as opposed to indulging the Irascible appetite) \\
\hline
HOPE & Integer (Negative being Despair Positive Hope) & Governs Irascible natural Hope \\
\hline
DARING & Integer (Negative being fear, Positive Daring) & Governs Irascible Courage \\
\hline
ANGER & Integer(Negative being anger and postive confidence) & Governs Anger\\
\hline


\end{longtable}

\begin{longtable}{| p{6cm} | p{3cm} | p{10cm} |}
\caption{Story Event Structure besides Character Variables}
\label{tab:StoryEvent}\\
\hline 
\textbf{Name} & \textbf{Data Type} & \textbf{Purpose} \\
\hline
id & String & Identifier inside of a dictionary which selects the event  \\
\hline
TYPE & String & Actual Grace Or Temptation?\\
\hline
PRECONDITIONS & SubObject & Holds Minimal Preconditions of the Character\\
\hline
POSTCONDITIONS\_ACCEPT & SubObject & Holds Character results on acceptance of grace (Either to accept Actual Grace or to Reject a Temptation\\
\hline
POSTCONDITIONS\_REJECT & SubObject & Holds Character results on rejection of grace (Either to reject Actual Grace or to fall into a Temptation\\
\hline
QUOTES\_SCRIPTURE & Boolean & Is scripture quoted at the end of the event as flavour text?\\
\hline
STATE\_OF\_GRACE & Boolean & How does this impact the character's state of grace?  \\
\hline
OUTPUT & String & Output on Post Condition \\
\hline
SCRIPTURE\_BANK & SubObject & Contains all of the Scriptural variations \\
\hline
VERSES & Array Of Strings & Contains all of the Scriptural verse blocks \\
\hline
SECOND\_CHARACTER & Boolean & Is there a second character involved? Also, this as a prefix to the virtue and passion statements indicate this character is referred to \\
\hline
THIRD\_CHARACTER & Boolean & Is there a third character involved? Also, this as a prefix to the virtue and passion statements indicate this character is referred to \\
\hline
\_UNDER & Suffix to a Virtue & Indicates the bound of a virtue or vice \\
\hline
\_OVER & Suffix to a Virtue & Indicates the bound of a virtue or vice \\
\hline
\_SECOND\_PERSON & Suffix to a Passion prior to UNDER/OVER & Indicates the person which the passion is related \\
\hline
\_THIRD\_PERSON & Suffix to a Passion prior to UNDER/OVER & Indicates the person which the passion is related \\
\hline
CONSEQUENT\_ACTIONS & Array of story action ids & Provides a set of randomly selected subsequent actions to be carried out\\
\hline
\end{longtable}


\begin{longtable}{| p{6cm} | p{3cm} | p{10cm} |}
\caption{Scripture Loading Class}
\label{tab:StoryEvent}\\
\hline 
\textbf{Name} & \textbf{Data Type} & \textbf{Purpose} \\
\hline

\end{longtable}
\section{Will Algorithms}
\subsection{Actual Grace}
If the total is lower than 60, then the grace is rejected
\begin{enumerate}
	\item Generate a random integer 0 and 100
	\item Add the summation of humility - pride
	\item Add the summation of prudence (Subvirtues - Subvices) 
	\item If it's against another person, add the passions of the relationship together
	\item Do the same with general passions
	\item Add the passions again if they're mentioned in the PRE\_CONDITIONS
	\item Add the sub virtues in the PRE\_CONDITIONS multiplied by 4
	\item Subtract the sub vices PRE\_CONDITIONS multiplied by 4
\end{enumerate}

\subsection{Temptation}
If the total is lower than 60, then the grace is rejected
\begin{enumerate}
	\item Generate a random integer 0 and 100
	\item Add the summation of humility - pride
	\item Add the summation of prudence (Subvirtues - Subvices) 
	\item If it's against another person, add the passions of the relationship together
	\item Do the same with general passions
	item Add the passions again if they're mentioned in the PRE\_CONDITIONS
	\item Add the sub virtues in the PRE\_CONDITIONS multiplied by 4
	\item Subtract the sub vices in the PRE\_CONDITIONS multiplied by 4
\end{enumerate}

\section{MOEA}

\section{Markov Chain Name Generator}

\section{LaTeX Output?}

\end{document}
